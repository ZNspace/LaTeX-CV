%-------------------------------------------------------------------------------
%	SECTION TITLE
%-------------------------------------------------------------------------------
\cvsection{项目经历}


%-------------------------------------------------------------------------------
%	CONTENT
%-------------------------------------------------------------------------------
\begin{cventries}


%---------------------------------------------------------
  \cventry
  {使开发更方便高效的全流程管理系统} % Role
  {AOS 构建发布系统} % Event
  {马蜂窝} % Location
  {2020. 03} % Date(s)
  {
    \begin{cvitems} % Description(s)
      \item {基于 Webhook 协议,接入企业微信,实现关键事件通知}
      \item {负责系统日常开发维护}
    \end{cvitems}
  }

% %---------------------------------------------------------
  \cventry
  {疫情期间 DAU 增长活动} % Role
  {晒晒我的下一站旅程} % Event
  {马蜂窝} % Location
  {2020. 03} % Date(s)
  {
    \begin{cvitems} % Description(s)
      \item {使用 CreateJS 完成动画交互}
      \item {由于疫情期间 DAU 惨淡, 需要快速尝试拉活,时间紧迫,故在一周内完成本活动开发}
      \item {上线后日均 pv 达到 2w+}
    \end{cvitems}
  }

%---------------------------------------------------------
  \cventry
  {与马来西亚机场合作,增加境外用户流量入口,通过小程序的形式,使用户了解马蜂窝,增加用户认知} % Role
  {马来西亚机场小程序} % Event
  {马蜂窝} % Location
  {2020. 01} % Date(s)
  {
    \begin{cvitems} % Description(s)
      \item {基于 mpvue 开发的小程序,使用内部拼装平台构建发布}
      \item {负责页面拆分,抽象出可复用组件}
      \item {npm 组件库维护}
    \end{cvitems}
  }


%---------------------------------------------------------
  \cventry
  {为管理内部运营账号搭建的平台,提供了运营账号申请、审批、创建、管理、统计相关的功能} % Role
  {运营账号管理后台} % Event
  {马蜂窝} % Location
  {2019. 09} % Date(s)
  {
    \begin{cvitems} % Description(s)
      \item {负责系统原型及页面 UI 设计}
      \item {负责项目选型及路由权限设计}
    \end{cvitems}
  }

%---------------------------------------------------------
  \cventry
  {针对活动项目分散、可维护性低的问题,对老活动项目进行整合、性能调优及架构升级} % Role
  {移动端活动脚手架} % Event
  {马蜂窝} % Location
  {2019. 07} % Date(s)
  {
    \begin{cvitems} % Description(s)
      \item {使用最新版本的 Webpack 4及babel 7构建}
      \item {合并项目,选用 Multi Entry 架构,将多个独立的老项目抽象,变成新项目下的入口,便于维护升级}
      \item {为多入口项目提供启动插件,加快本地构建速度、提高开发效率}
    \end{cvitems}
  }

%---------------------------------------------------------
  \cventry
    {对用户产生内容(UGC)进行安全审查的任务式审核系统,可承载多种审核内容类型,包括常规文字内容、图片内容、视频内容、用户注册信息等} % Role
    {内容安全审核系统} % Event
    {马蜂窝} % Location
    {2019. 03} % Date(s)
    {
      \begin{cvitems} % Description(s)
        \item {负责系统各功能模块开发以及需求迭代}
        \item {技术栈选用 Vue + Vuex + Element UI}
        \item {打造全类型审核平台,为审核团队赋能}
      \end{cvitems}
    }

%---------------------------------------------------------
  \cventry
    {可视化定义接口文档,mock 数据管理平台,解决了前后端分离场景中开发依赖、接口定义不明确等问题} % Role
    {接口管理平台} % Event
    {马蜂窝} % Location
    {2018. 11} % Date(s)
    {
      \begin{cvitems} % Description(s)
        \item {负责项目架构设计及开发,技术选型上前端使用 Vue 全家桶;后端使用 Node.js,选用 Nest.js, TypeORM 开发}
        \item {平台提供接口可视化定义、文档、接口 mock、操作日志等功能}
        \item {平台提供 Webpack plugin,降低项目接入成本}
        \item {在研发部门推广使用,接入项目超过50个,每周有上万次 mock}
      \end{cvitems}
    }

%---------------------------------------------------------
  \cventry
    {微信服务号及移动端h5站点,为客户提供移动端交易的平台} % Role
    {嘉实财富 Club} % Event
    {嘉实基金} % Location
    {2018. 03} % Date(s)
    {
      \begin{cvitems} % Description(s)
        \item {前端负责人,带领3人团队从零构建}
        \item {负责前端技术方案选型,模块划分,项目管理}
        \item {使用 React + Dva + Antd Mobile 组件化开发}
      \end{cvitems}
    }

%---------------------------------------------------------
\end{cventries}
